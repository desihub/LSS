\documentclass[11pt]{article}

\usepackage[top=1in, bottom=1in, left=1in, right=1in]{geometry}
\usepackage[square,numbers]{natbib}
\usepackage{graphicx}
\usepackage{float}
\usepackage{bm}
\usepackage{wrapfig}
\usepackage{subcaption}
\usepackage{hyperref}

\begin{document}

\begin{center}
\Large{\textbf{Large-scale structure catalogue and random generation task force}}
\end{center}

The charge of this task force is advance the development of tools for creating
LSS catalogue and associated random catalogues. The work is divided into three
tasks
\begin{itemize}
\item LSSIII: Assessment of the data model for LSS catalogue (lead Mariana Vargas-Magana, mmaganav@fisica.unam.mx)
\item LSSI: Advancing the LSS catalogue generation code (lead Lado Samushia, lado@phys.ksu.edu)
\item LSSII: Constructing tools for random catalogue generation (Angela Burden, angela.burden@yale.edu)
\end{itemize}

\subsection*{LSSIII} The goal is to define data model for the output LSS
catalogue and random catalogues. This group will interact with galaxy
clustering, bright galaxy survey, and $C^3$ working groups to make sure that all
of the information that's required for subsequent science analysis is available
in LSS and random catalogues in a well-documented and easy to use manner.

\subsection*{LSSI} The goal is to develop a code that produces LSS catalogue
compliant with the data model defined by \textit{LSSIII}. We have to understand
where all of the information is coming from and what are the relevant data
models. Some of the information (e.g., local density, etc.) may have to be
computed from scratch from a more basic data.

\subsection*{LSSII} The goal is to develop a set of tools that produce random
catalogues compliant with the data model defined by \textit{LSSIII}. We have to
understand all density variations that are of non-cosmological origin and make
sure that they are reflected in either the LSS or the random catalogues.

\section{LSSIII}

The data model for the LSS catalogue is currently not fully defined. Below is a
provisional list of quantities that should go to the LSS catalogue. One of the
goals for the Argonne meeting is to review this list and make sure nothing
important is missing.

\begin{enumerate}
\item Basic information
\begin{itemize}
\item right ascension
\item declination
\item redshift
\item object type
\end{itemize}
\item Imaging information
\begin{itemize}
\item Identifier (Brick ID, \ldots)
\item Where is the imaging coming from (DECaLS, WISE, \ldots)
\item Fluxes in various bands
\item Exposure information
\item Observational conditions
\item Quality information
\end{itemize}
\item Targeting information
\begin{itemize}
\item Target ID
\item Target completeness
\item Target priority
\end{itemize}
\item Tiling information
\begin{itemize}
\item Observing tile IDs and fiber IDs
\item List of objects that could have been assigned the same tile/fiber
combination
\item Observed by which program
\item Tile/fiber combinations that could have observed this object
\item Likelihood that the object is assigned a fiber (defined over multiple 
realisations of a fiber assignment code)
\end{itemize}
\item Spectroscopic information
\begin{itemize}
\item Redshift error
\item Redshift failure
\item Warning flags
\item Likelihood of observing this redshift/line strength combination
\end{itemize}
\item Auxiliary information
\begin{itemize}
\item weights to describe systematic effects and incompletenesses
\item weights to enhance clustering signal (FKP, PVP, \ldots)
\end{itemize}
\end{enumerate}

\section{LSSI}

The main objective is to identify where all of the information listed in the
previous section will come from and develop a code that collects (or computes)
all this information.
\begin{enumerate}
\item  The \textit{ra} and \textit{dec} will come from the imaging and will be
propagated all the way (?); Not likely to change throughout the pipeline.
Redshift will be measured by the spectroscopic pipeline. Object type will be
assigned by the target selection pipeline; This may occasionally change after
redshifts are taken so will have to be taken from spectroscopic pipeline output.
\item Will be read from imaging data files. This information is not going to
change down the pipeline. The DR1 of DECaLS data is available and documented in
\url{http://legacysurvey.org/dr1/}.
\item This will come from the target selection pipeline. Some of the information
(e.g., target type) may change and will have to be retrieved after spectroscopy
takes place. The target selection code is being implemented here:
github.com/desihub/desitarget.
\item The information will come from the output of the fiber assignment code.
Some of the entities here are not well defined. For example, it's not clear that
the likelihood that an object is assigned a fiber can be computed easily on
practice, or that it will be useful in mitigating fiber assignment effects. The
fiber assignment codes can be found here:
\url{https://github.com/desihub/fiberassign_sqlite} and
\url{https://github.com/desihub/fiberassign}. The data model for the fiber assignment
output is not defined at the moment.
\item This information will come from the spectroscopic pipeline. The data model
is defined and can be found here:
\url{https://github.com/desihub/desidatamodel/blob/master/doc/DESI_SPECTRO_REDUX/PRODNAME/bricks/BRICKNAME/zbest-BRICKNAME.rst}.
Where does the likelihood of measuring a redshift come from? Is this information
better put in the random catalogue by pushing the random catalogue through the
spectroscopic pipeline?
\item There are two types of weights. The ones that correct for systematics and
the ones that are meant to increase the signal to noise. We will discuss this in
Argonne. Some of the systematic are more naturally reflected in randoms rather
than as weights in the LSS catalogue. The S/N weights will depend on local
number density and bias estimate and will have to be computed from scratch.
\end{enumerate}

\section{LSSII}

The main objective is to develop a set of tools for creating random catalogues.
The random catalogues should reflect all the variations in number density that
are not of cosmological origin. All parts of pipeline (imaging, targeting,
tiling, fiber assignment, spectroscopic) will imprint on the mask and selection
function. Some of this are better described as weights of objects in LSS
catalogue, and some of them as density of points in randoms. We have to decide
in Argonne what goes where. 

The idea of ``forward modelling'' random catalogues is very popular. The idea is
to start from a uniform distribution of objects and push the randoms through the
same data pipeline. The hope is that imaging, spectro, tiling, will imprint the
same systematics in randoms, making it possible to remove them from the
clustering measurements. This has been done successfully for tiling/fiber
assignment, and it feels like this should work for the imaging and spectroscopic
pipelines as well.

There are some effects that may not be correctable in either LSS catalogue or
randoms. The incompleteness in fiber assignment due to local density of sources
is possibly one such effect. If that's the case, they will have to be modelled
at the analysis stage rather than corrected in data.

\section{What do we want to accomplish at Argonne meeting \& Issues to be
addressed}

\begin{itemize}
\item Define provisional data model for the output of LSS and random catalogues.
Put documentation in github.
\item Understand the format of imaginglss. Talk/tutorial by someone involved in
imaginglss?
\item Understand the format of target selection database. What can change down
the pipeline? Talk/tutorial from a TS person?
\item Understand data model for fiber assignment output. What information do we
want to put in the LSS catalogue exactly? What does a probability of an object
to have a fiber assigned mean?
\item Understand data model for spectroscopic pipeline. 
\item What kind of auxiliary information should go into LSS catalogue?
\item What kind of systematic effects go into LSS catalogue as weights and what
goes into random catalogue as completeness?
\item Is full forward modelling of randoms feasible? Can we push uniform
catalogues through the whole pipeline (imaging, TS, fiber assignment, spectro)?
Are there easier ways of accounting for some of the effects?
\item What to do with the overdensity dependent effects (overdensity dependence
of fiber assignment efficiency)? Can they be removed from the data? If not, what
information should be propagated so that the effects can be modelled at the
analysis stage?
\item Survey available effort. Identify codes and modules to be written.
Timeline for next six months.
\end{itemize}

\section{Agenda for Argonne meeting}

\begin{itemize}
\item \textbf{Mon 2 parallel session} General overview of the task-force.
Discussion about the LSS and random catalogue data model. Are there any
obviously missing pieces. In what way will the BGS and dark time catalogues
differ? Does $C^3$ require it's own catalogue?
\item \textbf{Mon 3 parallel session} A talk/tutorial from the imaginglss
person. What will LSS and random generating codes require? What codes are
available (e.g., a script generating randoms for imaging)? What part of imaging
incompleteness goes to the LSS and what goes what goes to randoms?
\item \textbf{Tue 1 parallel session} A talk/tutorial from a TS person. Where
does the incompleteness information go? What will LSS and random generation
require exactly?
\item \textbf{Tue 2 parallel session} A talk/tutorial from a fiber assignment
person. What will LSS and random generation require? Does FA output data model
have everything that we need? How do we describe FA incompleteness?
\item \textbf{Tue 3 parallel session} A talk/tutorial from a spectro person.
Understand data model for the spectro pipeline output. Can a redshift, type and
line strength dependent weight be defined? What will LSS and random generating
codes need and is it accommodated in the data model?
\item \textbf{Wed 1 parallel session} Define a provisional data model for the
output of the LSS and random catalogues. Put the data model and documentation in
github. Create sample files. Identify person(s) who will do this.
\item \textbf{Wed 2 parallel session} Finalize a high-level description of the
LSS catalogue generating code describing modules and data flow. Identify people
who will write specific components of the code.
\item \textbf{Wed 3 parallel session} Finilize a high-level description of the
random generating code describing modules and data flow.  Identify people who
will write specific components of the code.
\item \textbf{Thu 1 parallel session} General overview. Create a rough timeline
for next six months.
\end{itemize}
\end{document}
